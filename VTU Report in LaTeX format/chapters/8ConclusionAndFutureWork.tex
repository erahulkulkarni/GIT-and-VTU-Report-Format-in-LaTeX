\chapter*{Conclusion and Future work}
\addcontentsline{toc}{chapter}{Conclusion and Future work}

This project provides a comprehensive study of collective role of ECM confinement, 
cell motility and ECM proteolysis on hierarchical theory based tumor development mediated by CSC.
In existing systems CA have been modelled to explore CSCs proliferation as factor of only oxygen diffusion, motility or ECM proteolysis.
Leading to proposed system which overcomes the limitations by integrating all of ECM architecture, motility and ECM proteolysis in one single CA model.\\\

\noindent While current study provides many new insights about CSC driven tumor development and generation of intra-tumor heterogeneity. 
Major conclusions that can be drawn from results are CSC symmetric division rate fosters tumor growth and controls intra-tumor cellular heterogeneity, 
ECM confinement can check tumor growth in absence of ECM proteolysis, 
Increased cell motility facilitates cancer progression confined ECM and enhances TAC content of tumor population, 
Cell motility and ECM proteolysis suppresses the effect of ECM confinement and enhances tumor progression. \\\

\noindent
There are several aspects like BCs properties like contractility, higher sensing radius and ECM cross linking not taken into account in this work 
and requires further extension of this model. Contractility influences motility of biological cells, 
higher sending radius would enlarge neighbourhood to be considered for ES and BC properties and 
ECM cross linking in fibers gives additional stiffness to ECM.
Understanding the role of mutation and other cellular transformation processes like EMT are the future directions of this work.
