\chapter{System Requirements}
\section{Introduction}
System requirements cover all of the requirements at the system level, which describe the functions the CA as a whole should fulfil to satisfy the requirements. 
It is expressed in an appropriate combination of textual statements, views and non-functional requirements; 
the latter expressing the levels of speed, reliability, scalability and standards to meet.
\section{Software requirements}
The softwares required to implement this project can be installed using sudo apt-get command or through Ubuntu Software Center, 
and usage of each has be mentioned in Table \ref{Tbl_softwares_used}
  \begin{table}[H]
	\begin{center}
		\begin{tabular}{ |c | c | }
			\hline
			\textbf{Software} & \textbf{Use}  \\ \hline
			  Ubuntu 14.04	 & 	Operating System	\\  \hline
			  C++ 		 &      Programming language to code biological properties and transitions.\\  \hline
			  Code::Blocks	 & 	IDE for C++ programming language	\\  \hline
			  GNU Octave	 & 	Image generation	\\  \hline
			  ImageJ	 & 	Video generation	\\  \hline
			  RStudio	 & 	Plotting results	\\  \hline
			  MATLAB$\circledR$	 & 	Result quantification	\\  \hline
			  QtiPlot	 & 	Plotting results	\\  \hline
			  OpenOffice	 & 	Save quantifications	\\  \hline
		\end{tabular}
		\caption{Softwares used}
		\label{Tbl_softwares_used}
	\end{center}
\end{table}

\section{Hardware requirements}
Hardware requirements define the minimal and optimal configurations for the system ~\cite{SoftwareEngineering}. 
Following are the hardware requirements for this project
\begin{table}[H]
	\begin{center}
		\begin{tabular}{ c  c  c  }
			  Processor	 & : &	1.5 GHz	\\
			  Physical memory 		 & : &      2 GB\\
			  Secondary memory	 & : &	47 GB	\\
		\end{tabular}
	\end{center}
\end{table}


\section{Functional requirements}
These are statements of services the system should provide, 
how the system should react to particular inputs, and how the system should behave in particular situations.
In some cases, the functional requirements may also explicitly state what the system should not do ~\cite{SoftwareEngineering}. 
In this project functional requirement describes the biological properties, constraints and operations that have to be satisfied by Mathematical Model, 
which should simulate biological cell division, spread of CSC as the function of intrinsic parameters, ECM, motility and proteolysis.
  
\section{Non-functional requirements}
Non Functional Requirements describe the constraints on development process and standards to be followed in development process.
 \begin{enumerate}
  \item Speed - on mentioned hardware requirement one simulation should run in around 5 minutes.
  \item Reliability - simulation results should not vary drastically.
  \item Scalability - simulation can be scaled to size of 10000 * 10000.
  \item camelCase used for function names
 \end{enumerate}  
 
 \begin{list}{}{}
  \item Empty List
 \end{list}

 
\section{Conclusion}
The software, hardware, functional and non-functional requirements for the project was elicited covering various open source softwares, 
hardware requirements and nonfunctional requirements covering scalability, response time and biological constraints to be followed.


