\documentclass[17pt]{beamer} 
\usepackage{amsmath}
\definecolor{blue}{RGB}{76,0,153}
\setbeamercolor{alerted text}{fg=blue}
\definecolor{blue}{RGB}{76,0,153}
\setbeamercolor{structure}{fg=blue}
\beamertemplateshadingbackground{white!8}{blue!7}
\usepackage{beamerthemesplit}
%\usepackage{beamerthemeshadow}
%\setbeamersize{text margin left=0.25cm,text margin right=0.5cm}

\logo{\includegraphics[height=1cm]{st-logo-new.png}}

\begin{document}
\sffamily \bfseries
 \title
[Introduction to Biopython\hspace{0.75cm}]
{Introduction to Biopython}
\author
[Snehalatha]
{\small Talk to a Teacher \\ http://spoken-tutorial.org\\ National Mission on Education
  through ICT \\ http://sakshat.ac.in \\ [0.2cm]
 Script: Snehalatha Kaliappan\\IIT Bombay \\ [0.2cm]
{\small 15 May 2015} \\[0.2cm]
}
\begin{frame}
\titlepage
\end{frame}

\begin{frame}
\frametitle{Learning Objectives} \pause
\begin{itemize}[<+-|alert@+>]
\item Important features of Biopython
\item Download and installation for linux OS
\item Translation of a DNA sequence to a protein sequence using Biopython tools
\end{itemize}
\end{frame}

\begin{frame}
\frametitle{Pre-requisites} \pause
\begin{itemize}[<+-|alert@+>]
\item Familiar with Undergraduate Biochemistry or Bioinformatics
\item Basic Python programming
\item Refer to Python Spoken Tutorials at {\color{blue}http://spoken-tutorial.org} 
\end{itemize}
\end{frame}

\begin{frame}
\frametitle{System Requirements} \pause
\begin{itemize}[<+-|alert@+>]
\item Ubuntu OS version 12.04
\item Python version 2.7.3
\item IPython interpreter version 0.12.1
\item Biopython version 1.58
\end{itemize}
\end{frame}

\begin{frame}
\frametitle{About Biopython} \pause
\begin{itemize}[<+-|alert@+>]
\item Biopython is a collection of modules for computational biology
\item It can perform most basic to advanced tasks required for bioinformatics
\end{itemize}
\end{frame}

\begin{frame}
\frametitle{Biopython functionality for bioinformatics} \pause
\begin{itemize}[<+-|alert@+>]
\item Parsing (Extracting information) from various file formats ({\color {blue}FASTA, Genbank} etc)
\item Download data from  database websites ({\color {blue}NCBI, ExPASY}) 
\item Run bioinformatic algorithms such as BLAST
\end{itemize}
\end{frame}

\begin{frame}
\frametitle {Biopython functionality for bioinformatics } \pause
\begin{itemize}[<+-|alert@+>]
\item Tools for performing common operations on sequences (complements, transcription, translation etc)
\item Code for dealing with alignments
\item Code to split up tasks into separate processes 
\end{itemize}
\end{frame}

\begin{frame}
\frametitle{Download}\pause
\begin{itemize}[<+-|alert@+>]
\item Biopython package is not part of Python distribution, so it  needs to be downloaded independently
\item {\color {blue} http://biopython.org/wiki/Download} 
\end{itemize}
\end{frame}

\begin{frame}
\frametitle{Installation on Linux OS}\pause
\begin{itemize}[<+-|alert@+>]
\item Install Python, IPython and Biopython packages using Synaptic Package Manager
\item Prerequisite software will be installed automatically
\item Additional packages must be installed for graphic outputs and plots
\end{itemize}
\end{frame}


\begin{frame}
\frametitle{Task}
\begin{itemize} 
\item (1) Create a sequence object for the coding DNA strand
\item (2) Transcription of  coding DNA strand to mRNA
\item (3) Translation of mRNA to a protein sequence
\end{itemize}
\end{frame}

\begin{frame}
\frametitle{Sequence Object}
\begin{itemize} 
\item {\color {blue}`ATGTTACACTCCCGATGA'}
\item Create a sequence object for the above coding DNA strand
\end{itemize}
\end{frame}

\begin{frame}
\frametitle{Summary}
\begin{itemize} 
\item Important features of Biopython 
\item Download and installation on Linux OS
\item Convert the DNA sequence into sequence object
\end{itemize}
\end{frame}

\begin{frame}
\frametitle{Summary}
\begin{itemize} 
\item Transcription of the DNA sequence to mRNA ({\color {blue}`transcribe'} method) 
\item Translation of mRNA to protein sequence ({\color{blue} `translate'} method)
\end{itemize}
\end{frame}

\begin{frame}
\frametitle{Assignment}\pause
\begin{itemize}
\item Translate the DNA sequence to a protein sequence
\item {\small \color {blue} `ATGGCCCTATAGTGTCTAAGCTAG'}
\item The output shows an internal stop codon 
\item Translate the DNA strand till first in frame stop codon  
\end{itemize}
\end{frame}

\begin{frame}
\frametitle{About the Spoken Tutorial Project}
\begin{itemize}
\item Watch the video available at {\color{blue} http://spoken-tutorial.org /What\_is\_a\_Spoken\_Tutorial}
\item It summarises the Spoken Tutorial project \pause
\item If you do not have good bandwidth, you can download and watch it
\end{itemize}
\end{frame}

\begin{frame}
\frametitle{Spoken Tutorial Workshops}
The Spoken Tutorial Project Team
\begin{itemize}
\item Conducts workshops using spoken tutorials
\item Gives certificates to those who pass an online test
\item For more details, please write to {\color{blue} contact@spoken-tutorial.org}
\end{itemize}
\end{frame}

\begin{frame}
\frametitle{Acknowledgements}
\begin{itemize}
\item Spoken Tutorial Project is a part of the Talk to a Teacher project
\item It is supported by the National Mission on Education through
  ICT, MHRD, Government of India
\item More information on this Mission is available at
{\color{blue} http://spoken-tutorial.org /NMEICT-Intro}
\end{itemize}
\end{frame}

\end{document}


