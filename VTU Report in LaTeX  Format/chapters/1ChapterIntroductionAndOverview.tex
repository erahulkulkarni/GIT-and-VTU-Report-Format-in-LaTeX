\chapter{Introduction and Overview}
\pagenumbering{arabic}

\section{Introduction}
This chapter gives introduction to biological concepts, Cellular Automata(CA), motivation behind the project, list of existing systems and their limitations, the proposed system and organization of report.

\section{Biological concepts}

%\paragraph{}
Various tissues, organs have different stiffness. Tissues are not made up solely of biological cells.
A substantial part of their volume is extracellular space, which is largely filled by an intricate network of macromolecules constituting the Extracellular Matrix(ECM). 
This matrix is composed of a variety of proteins and polysaccharides that are secreted locally and assembled into an organized meshwork in close association with the surface of the cell that produced them.
The ECM can influence the organization of a cell's cytoskeleton. This can be vividly demonstrated by using transformed (cancer like) fibroblasts in culture. 
Transformed cells often make less fibronectin than normal cultured cells and behave differently.~\cite{MolecularBiologyOfTheCell}.


~\\A FSA is a machine that has finite set of states the machine can be in, reads input from a set of characters and transition occurs on condition of present state and input, 
a FSA is a restricted Turing machine~\cite{FiniteStateMachine}.
Biological systems, having cell types and transitions between cell types based on cell and neighbourhood properties, can be modelled with FSA which can represent cell type change as state changes in FSA and cell and neighbourhood properties as input to FSA.
This project considers four type of cells CSC, Transiently Amplifying Cell (TAC), Terminally Differentiated Cell (TDC), ECM Site (ES) as states of Automata, 
where CSC, TAC and TDC are considered BCs. Transition function is dependent on BCs and ES properties.

\section{Motivation behind the project}
CSCs have been shown to associate with different aggressive cancer phenotypes including drug resistance.
CSCs are transformed cancer cells possessing the properties similar to stem cells. 
The project will contribute to our understanding of how CSCs contribute to cancer invasiveness.


\section{Existing system}
\label{ExistingSystem}
Existing systems have probed CSC dynamics as function of oxygen diffusion or only motility of BCs.
Generation of intratumor heterogeneity as function of intrinsic motility of cell, exact role of motility has not been examined.
The effects of increase in ECM confinement to tumor growth and cellular heterogeneity within tumor is not probed.
Tumor progression and emergence of intratumor heterogeneity as function of ECM proteolysis is not well understood.

\section{Proposed system}
The proposed system integrates ECM confinement, BC motility and proteolysis propensity to explore the lacks existing system mentioned in ~\ref{ExistingSystem}.
This project develops a Computational Model (CM) that combines discrete and continuous modelling approaches.

\section{Organization of report}
This thesis is divided into various chapters and each chapter deals with a specific topic.
Chapter 2 contains the literature survey and the technologies that are needed for this work.
Subsequently, Chapter 3 addresses the System Requirements Specification. 
Chapter 4 presents the design of the system in terms of modules and the approach used. 
Chapter 5 shows the implementation details of each of the major modules and analysis drawn from implementation. 
Chapter 6 explains the system testing.
Chapter 7 explains results, type of files obtained from simulation.
Finally, the conclusion of the work and future enhancements is presented in Chapter 8.

\section{Conclusion}
This chapter covers biological concepts of CSCs, 
overview on CSCs proliferation implementation using CA and using it to extend to proposed system of CSCs proliferation as a function of ECM, 
biological cell motility and 
ECM proteolysis propensity.

